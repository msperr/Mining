\section{Physical Model}

\begin{frame}
	\frametitle{Lagrange Formalism}
	\onslide<1->
	Method to describe dynamics of an accelerated system\\
	\vspace{0.45cm}
	\begin{columns}
		\column{.5\linewidth}
		\column{.55\linewidth}
		%\begin{small}
		\onslide<2->
		\begin{tabular}{ll}
			& \\
			T & kinetic energy\\
			V & potentials\\
		\end{tabular}
		\onslide<3->
		\begin{tabular}{ll}
			F & non-conservative external forces\\
		\end{tabular}
		\onslide<4->
		\begin{tabular}{ll}
			r & points of actions of forces F\\ %position vector
			q & free variables\\
			Q & generalized forces\\
		\end{tabular}
		%\end{small}
	\end{columns}
\end{frame}

\begin{frame}
	\setbeamertemplate{blocks}[rounded][shadow=false]
	\frametitle{Lagrange Formalism}
	Method to describe dynamics of an accelerated system\\
	\vspace{0.45cm}
	\begin{columns}
		\column{.02\linewidth}
		\column{.43\linewidth}
		\centering
		\begin{block}{}
			\vskip -3mm
			\begin{align*}
				&\frac{d}{dt}\left(\frac{\partial T}{\partial \dot{q}_i}\right) -
				\frac{\partial T}{\partial q_i} +
				\frac{\partial V}{\partial q_i}
				= Q_i \\
				& Q = \left(\frac{\partial r}{\partial q}\right)^T F
			\end{align*}
			\vskip -3mm
			\hspace*\fill
		\end{block}
		\column{.05\linewidth}
		\column{.55\linewidth}
		%\begin{small}
		\begin{tabular}{ll}
			& \\
			T & kinetic energy\\
			V & potentials\\
			F & non-conservative external forces\\
			r & points of actions of forces F\\ %position vector
			q & free variables\\
			Q & generalized forces\\
		\end{tabular}
		%\end{small}
	\end{columns}
\end{frame}	

\begin{frame}
	\frametitle{Physical Model of Excavator}
	
	%left, bottom, right, top
	%\includegraphics[trim=22cm 5cm 2cm 23cm, clip=true, width=\linewidth]{img/Excavator_Only}
	
	\begin{columns}
		\column{.6\linewidth}
		\centering
		\includegraphics[trim=30cm 5cm 30cm 23cm, clip=true, width=\linewidth]{img/Excavator_Only}
		\column{.4\linewidth}
	\end{columns}
	
\end{frame}

\begin{frame}
	\frametitle{Physical Model of Excavator}
	
	%	invariant under transformation of coordinates $\rightarrow$ appropriate choice of 	coordinate system
	
	%first fix coordinate system $\rightarrow$ consider positions vectors, coordinates of mass points,... in this system\\
	
	%\includegraphics[trim=22cm 5cm 2cm 23cm, clip=true, width=\linewidth]{img/Excavator_coordinates2}
	
	\begin{columns}
		\column{.6\linewidth}
		\centering
		\includegraphics[trim=30cm 5cm 30cm 23cm, clip=true, width=\linewidth]{img/Excavator_coordinates2}
		\column{.4\linewidth}
		coordinate system
	\end{columns}
	
\end{frame}

\begin{frame}
	\frametitle{Physical Model of Excavator}
	
	%think of degrees of freedom $\rightarrow$ angle and length side arm $\rightarrow$ forces in the system will depend on excavator configuration and therefore on $s$ and $\theta$\\
	
	%\includegraphics[trim=22cm 5cm 2cm 23cm, clip=true, width=\linewidth]{img/Excavator_dof2}
	
	\begin{columns}
		\column{.6\linewidth}
		\centering
		\includegraphics[trim=30cm 5cm 30cm 23cm, clip=true, width=\linewidth]{img/Excavator_dof2}
		\column{.4\linewidth}
		degrees of freedom
		\begin{itemize}
			\item{length $s$}
			\item{tilt angle $\theta$}
		\end{itemize}
	\end{columns}
	
	% s distance between attachment point and upper cable reel of green pulley
	
\end{frame}

\begin{frame}
	\frametitle{Physical Model of Excavator}
	\onslide<1->
	%\includegraphics[trim=22cm 5cm 2cm 23cm, clip=true, width=\linewidth]{img/Excavator_mass2}
	\begin{columns}
		\column{.6\linewidth}
		\centering
		\includegraphics[trim=30cm 5cm 30cm 23cm, clip=true, width=\linewidth]{img/Excavator_mass2}
		\column{.4\linewidth}
		
		movable centers of gravity of
		\begin{itemize}
			\item{shovel $M_1$}
			\item{arm $M_2$}\\
		\end{itemize}
		\onslide<2->
		
		\begin{small}
			\begin{align*}
				&\frac{d}{dt}\left(\frac{\partial T}{\partial \dot{q}_i}\right) -
				\frac{\partial T}{\partial q_i} +
				\frac{\partial V}{\partial q_i}
				= Q_i \\
				& Q = \left(\frac{\partial r}{\partial q}\right)^T F\\
			\end{align*}
		\end{small}
	\end{columns}
	
	%where are the point masses? only consider point masses which can be moved i.e. depend on configuration $\rightarrow$ derivative wrt degrees of freedom\\
	%shovel and center of mass of side arm\\
	%other arm is fixed and cant be moved
	
\end{frame}

%\begin{frame}
%\frametitle{Kinetic Energy T}
%
%Example: energies of $M_2$ 
%\begin{align*}
%&E_{\text{kin},M_2} = \frac{1}{2}\ M_2\ \| 
%v_{O,M_2}(s,\theta,\dot{s},\dot{\theta}) \|^2 \\
%&E_{\text{rot},M_2} = \frac{1}{2}\ I_{M_2}(s)\ \dot{\theta}^2 \\
%\end{align*}
%
%all kinetic energies:
%\begin{align*}
%T\ =\ \ & E_{\text{kin},M_1} + E_{\text{kin},M_2} + E_{\text{rot},M_2}  \\
%& + E_{\text{rot},B_1} + E_{\text{rot},B_2} + E_{\text{rot},P_1} + E_{\text{rot},P_2} \\
%\end{align*}
%\end{frame}

%\begin{frame}
%\frametitle{Potential V}
%
%Example: potential of $M_2$
%\begin{align*}
%& V_{M_2} = M_2\ g\ h(s,\theta) \\
%\end{align*}
%
%all potentials:
%\begin{align*}
%& V = V_{M_1} + V_{M_2} \\
%\end{align*}
%\end{frame}

%\begin{frame}
%\frametitle{Involved Forces}
%\begin{figure}[bth]
%\begin{center}
%%left, bottom, right, top
%\includegraphics[trim=22cm 5cm 2cm 24cm, clip=true, 
%width=\linewidth]{img/Excavator_Only1}
%\end{center}
%\end{figure}
%\end{frame}

\begin{frame}
	\frametitle{Kinetic Energy}
	\begin{columns}
		\column{.6\linewidth}
		\centering
		\includegraphics[trim=30cm 5cm 30cm 23cm, clip=true, width=\linewidth]{img/Excavator_Only}
		\column{.4\linewidth}
		\begin{itemize}
			\item{movement of mass}
			\item{rotation of cable reel}
		\end{itemize}
	\end{columns}
\end{frame}

%\begin{frame}
%\frametitle{Involved Forces}
%\begin{figure}[bth]
%\begin{center}
%%left, bottom, right, top
%\includegraphics[trim=22cm 5cm 2cm 24cm, clip=true, 
%width=\linewidth]{img/Excavator_Only2}
%\end{center}
%\end{figure}
%\end{frame}

\begin{frame}
	\frametitle{Potential Energy}
	\begin{columns}
		\column{.6\linewidth}
		\centering
		\includegraphics[trim=30cm 5cm 30cm 23cm, clip=true, width=\linewidth]{img/Excavator_Only}
		\column{.4\linewidth}
		gravitational potential energy
	\end{columns}
\end{frame}

%\begin{frame}
%\frametitle{Involved Forces}
%\begin{figure}[bth]
%\begin{center}
%%left, bottom, right, top
%\includegraphics[trim=22cm 5cm 2cm 24cm, clip=true, 
%width=\linewidth]{img/Excavator_Only3}
%\end{center}
%\end{figure}
%\end{frame}

\begin{frame}
	\frametitle{Generalized Forces}
	\begin{columns}
		\column{.6\linewidth}
		\centering
		\includegraphics[trim=30cm 5cm 30cm 23cm, clip=true, width=\linewidth]{img/Excavator_Only}
		\column{.4\linewidth}
		\begin{itemize}
			\item{torque on cable reels}
			\item{friction of cable reels}
		\end{itemize}
	\end{columns}
\end{frame}

%\begin{frame}
%\frametitle{Generalized Forces Q}
%
%Example: force at $A_2$:
%\begin{align*}
%&F_{A_2} = \left[ \frac{\tau_{B_2}}{r_{B_2}} - \mu_{B_2} 
%\frac{\dot{s}}{r_{B_2}} - \mu_{P_2} \frac{\dot{s}}{r_{P_2}} \right]
%\left(
%\begin{matrix}
%\cos(\theta) \\
%\sin(\theta) \\
%\end{matrix}
%\right) \\
%\end{align*}
%
%all generalized forces:
%\begin{align*}
%&Q_s = \left( \frac{\partial r_{A_1}}{\partial s} \right)^T F_{A_1} 
%+ \left( \frac{\partial r_{A_2}}{\partial s} \right)^T F_{A_2} \\
%&Q_{\theta} = \left( \frac{\partial r_{A_1}}{\partial \theta} 
%\right)^T F_{A_1} + \left( \frac{\partial r_{A_2}}{\partial \theta} 
%\right)^T F_{A_2} \\
%\end{align*}
%\end{frame}

%\begin{frame}
%\frametitle{Parameters}
%
%\begin{tabular}{ll}
%& \\
%masses & $M_1$, $M_2$ \\
%&\\
%inertia of pulleys & $I_{B_1}$, $I_{B_2}$, $I_{P_1}$, $I_{P_2}$ \\
%&\\
%friction coefficients & $\mu_{B_1}$, $\mu_{B_2}$, $\mu_{P_1}$, 
%$\mu_{P_2}$  \\
%\end{tabular}
%\end{frame}

\begin{frame}
	\frametitle{Physical Model of Excavator}
	
	Assumptions to the model:\\
	\vspace{0.6cm}
	\begin{itemize}
		\item no mass for the ropes
		\item shovel as point mass % attached on the side arm
		\item no slack / friction between ropes and cable reels
		% ropes and cable reel move with same velocity
		% only bearing friction (friction within cable reels)
	\end{itemize}
	
\end{frame}

\begin{frame}
	\frametitle{Lagrange Formalism}
	
	\begin{columns}
		\column{.5\linewidth}
		\begin{align*}
			&\frac{d}{dt}\left(\frac{\partial T}{\partial \dot{s}}\right) -
			\frac{\partial T}{\partial s} +
			\frac{\partial V}{\partial s}
			= Q_s \\
			&{}\\
			&\frac{d}{dt}\left(\frac{\partial T}{\partial \dot{\theta}}\right) -
			\frac{\partial T}{\partial \theta} +
			\frac{\partial V}{\partial \theta}
			= Q_{\theta} \\
		\end{align*}
		\column{.5\linewidth}
		\centering
		\includegraphics[trim=30cm 5cm 30cm 23cm, clip=true, width=\linewidth]{img/Excavator_ill_dof}
	\end{columns}
	
\end{frame}

\begin{frame}[c]
	\frametitle{Resulting ODE}
	\onslide<1->
	\vspace{-0.5cm}
	%2nd order ODE from Lagrange Formalism:
	\begin{align*}
		&A(x,p)
		\begin{pmatrix} 
			\ddot{s} \\ \ddot{\theta} \\
		\end{pmatrix}
		= b(x,u,p)
	\end{align*}
	
	\onslide<2->
	$\rightarrow$ Transformation into 1st order ODE
	\begin{align*}
		&\frac{d}{dt}
		\begin{pmatrix}
			s \\ \theta \\ \dot{s} \\ \dot{\theta}
		\end{pmatrix}
		=
		\begin{pmatrix}
			\dot{s} \\ \dot{\theta} \\ A^{-1}(x,p)b(x,u,p) \\
		\end{pmatrix} 
		= f(x,u,p) \\
	\end{align*}
	
	\vspace{-1.0cm}
	
	\begin{tabular}{ll}
		& \\
		state & $ x = (s,\theta,\dot{s},\dot{\theta})^T $ \\
		control & $ u = (\tau_1,\tau_2)^T $ \\
		parameters & $ p = (p_1,...,p_k)^T $ \\
	\end{tabular}
\end{frame}