\documentclass{article}

\usepackage[T1]{fontenc}
\usepackage[latin1]{inputenc}
\usepackage{lmodern}
\usepackage{amsmath,amssymb,amsthm}

\setlength{\parindent}{0pt}

\begin{document}

\section{Parameter Optimization}

\subsection{Parameters}

\paragraph{Examples}

\begin{itemize}
	\item{friction coefficients in the cable reels}
	\item{inertia of the arm and in the engine}
	\item{mass and mass distribution of the arm}
\end{itemize}

\paragraph{Why?} \hfill\\

Parameters are often hard to measure in reality. Control and motion are known on a real excavator. Thus, we identify the parameters this way. \\

Parameters can change in a long time usage, due to frequent temperature changes, abrasion and dirt. Then, one can measure the actual parameters only this way.

\subsection{Black Box}

For this purpose, we have received a real excavator model from Siemens. Since the content is valuable and complex, the model is a black box, implemented in MatLab. \\

Input:

\begin{itemize}
	\item{Control: Actual handling of the operator in the mine}
	\item{Parameters}
\end{itemize}

Output:

\begin{itemize}
	\item{Motion: Actual position of the excavator shovel over time}
\end{itemize}

For given trajectories of control and motion, we have to identify the parameters.

\subsection{Optimization}

Since we have no information about the black box model, we will have to use derivative-free optimization-methods. \\

For verifying our optimization procedure, we will use the model we have developed so far. For this model, we have all information and we can apply this as a reference, e.g. for error searching.

\end{document}