\documentclass[a4paper]{article}

\usepackage[T1]{fontenc}
\usepackage[latin1]{inputenc}
\usepackage{lmodern}
\usepackage{amsmath,amssymb,amsthm}

\setlength{\parindent}{0pt}
\pagestyle{empty}

\begin{document}

\section{Parameter Optimization}

After completing our own model, we focus on the second task, the identification of parameters for the excavator.

\subsection{Parameters}

\paragraph{Examples}

\begin{itemize}
	\item{Friction coefficients in the engine and in the cable reels}
	\item{Mass and mass distribution of the arm}
	\item{Inertia of the arm and in the engine}
\end{itemize}

\paragraph{Why?} \hfill\\

Parameters are often hard to measure in reality. Control and motion are known on a real excavator. Thus, we identify the parameters this way. \\

Parameters can change in a long time usage, due to frequent temperature changes, abrasion and dirt. Then, one can measure the actual parameters only this way.

\subsection{Black Box}

For this purpose, we have received a real excavator model from Siemens. Since the content is confidential and complex, the model is a black box, implemented in MatLab. \\

Input:

\begin{itemize}
	\item{Control: Actual handling of the operator in the mine}
	\item{Parameters}
\end{itemize}

Output:

\begin{itemize}
	\item{Motion: Position and torques of the excavator shovel over time}
\end{itemize}

For given trajectories of control and motion, we have to identify the parameters.

\subsection{Parameter Optimization}

We are optimizing the parameters for the black box model and our own model.

\begin{itemize}
	\item{Own model: all information, including derivatives, available}
	\item{Black box model: no information about the black box model \\
		$\Rightarrow$ Derivative-free optimization methods}
\end{itemize}

For given trajectories of control and motion, we define penalty terms for the deviation from the desired motion \\

$\Rightarrow$ minimizing penalty terms in order to achieve solutions for parameters.

\end{document}